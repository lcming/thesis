\chapter{Conclusions and Future Work}
In this work, we proposed an efficient, fast and flexible DSP core design, DeAr,
which manipulates SMT in a VLIW-fashion datapath.
The compact hardware design endows DeAr good scalability of multi-core architecture, 
which is suitable for the HSA programming model.
% which out-performs 
To demonstrate the compatibility of DeAr with HSA,
we also presented an architectural framework of the HSA platform with a multi-core DeAr DSP, 
and developed the complete compiler software.
\\\indent
The experiment result showed that
% though DCT is not well, matrix arithmetic is good
\\\indent
% pipeline
Possible future works include extending the DeAr ISA.
Although the current ISA are already capable of handling most DSP kernel benchmarks, 
we still preserve some function code and bit fields to offer the abundant freedom of enhancements, 
such as adding special function units and supporting interrupt instructions.
The user can easily develop new features in DeAr to satisfy the demands from applications.
Another promising work is conducting the full system emulation of DeAr on existing HSA emulators such as HSAemu~\cite{hsaemu}.
With the evaluation at the system level, the designer is more likely to achieve the better memory hierarchy and interconnection design for multi-core DeAr.
Beyond DSP, we are also exploring the possibilities of applying DeAr to GPU.
The efficient arithmetic allows DeAr to become a potential alternative to a conventional RISC shader core.
However, GPU features massive threads that context-switch frequently.
Consequently, a state-of-the-art GPU is usually equipped with the dedicated hardware scheduler~\cite{fermi} to manage them.
Applying such a hardware-based thread scheduler to DeAr still needs more studies.


