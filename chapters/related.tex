\chapter{Related Work}
    In the stream of DSP design, many studies have been conducted on register file (RF) organization as it becomes the dominating factor of cycle time, power consumption and chip area \cite{register}.
    Rixner \textit{et al.} indicated that, for conventional centralized organization, the cost grows significantly with the number of functional units (FUs) \cite{register}, posing a serious challenge to VLIW DSP design.
    A straightforward solution to it is partitioning the centralized RF into several parts, each of which serves specific FUs, such as \cite{cluster}.
    Studies like \cite{synzen} and \cite{dsplite} went even further by discarding the centralized RF and allocating each FU a dedicated one, which is also known as distributed RF organization.
    However, dividing RF leads to more complicated compiler design and overhead of inter-RF data transport.
    In addition, above approaches still suffer from poor code density, which is an inherent problem from the VLIW architecture.
    \\\indent
    Some approaches avoided the aforementioned cost growth by customizing the data-path and limiting the number of ports on RF.
    Ou \textit{et al.} proposed composite FUs for DSP, which cascade FUs with a specific order and demand fewer ports on RF \cite{cascade} \cite{hearaid}.
    Similar techniques are often applied to ASIP design but they usually lack flexibility to target general purposes.
    There are further researches that achieve power reduction on RF by minimizing the number of accesses to it.
    Chen \textit{et al.} proposed a simulated-annealing based scheduling that aggressively forwards data from FU outputs to inputs instead of accessing RF \cite{multistage}.
    \cite{move} presented a processor framework: MOVE, which features the separation of data transport and operation in the data-path. 
    The user can program the bypass network in MOVE, and avoid most of accesses to RF by clever data transport.
    Such an architecture that manipulates data transport usually refers to transport-triggered architecture (TTA).
    The Work from Lama \textit{et al.} \cite{ttagpu} also demonstrated a framework that takes the advantage of TTA in graphic processing unit (GPU), and this idea is a potential alternative for DSP design.
    \\\indent
    On the other hand, one may notices that none of the above implementations adopts HSA \cite{systemspec}, which is a promising standard for embedded DSP platforms.
    So far, most of studies on HSA focus on the integration of CPU and GPU. \cite{hsaemu} presented a full system emulator for HSA platforms that include CPUs and GPUs.
    Beyond emulation, \cite{hsacyc} further illustrated a cycle-accurate HSA simulator that integrates QEMU \cite{qemu} and GPGPU-sim \cite{gpgpusim}.
    To the best of our knowledge, little or no research has been performed to apply the standard to DSP platforms.



