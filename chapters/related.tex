\chapter{Related Work}
\label{cha:related}
    In the stream of DSP design, many studies have been conducted on RF organization since it became the dominating factor of clock cycle, power dissipation and chip area \cite{register}.
    Rixner \textit{et al.} indicated that, for conventional centralized organization of RF, the cost grows significantly with the number of functional units (FUs) \cite{register}, posing a severe challenge to VLIW DSP design.
    A straightforward solution to it is partitioning the centralized RF into several parts, each of which serves specific FUs, such as \cite{cluster}.
    Studies like \cite{synzen} and \cite{dsplite} went even further by discarding the centralized RF and allocating each FU a dedicated one, which is also known as distributed RF organization.
    However, dividing RF leads overhead of inter-RF data transport and complicates the compiler design.
    In addition, the above solutions still suffer from poor code density, which is an inherent problem from the VLIW architecture.
    \\\indent
    Some approaches avoided the aforementioned cost growth by customizing the datapath and limiting the number of ports in RF.
    Ou \textit{et al.} proposed the composite-ALU DSP, which cascades FUs with a specific order and demand fewer ports in RF \cite{cascade} \cite{hearaid}.
    Similar techniques are often applied to ASIP design but they usually lack flexibility to target general purposes.
    There are further researches that achieve power reduction in RF by minimizing its access frequency.
    Chen \textit{et al.} proposed a simulated-annealing based scheduling for DSPs which aggressively forwards data from accumulators to FU inputs instead of accessing RF every cycle\cite{multistage}.
    \cite{move} presented a processor framework: MOVE, which features the separation of data transport and operation in the datapath. 
    The user can program the bypass network in MOVE, and thus avoid redundant accesses to RF by clever data transport.
    Such an architecture that manipulates data transport is usually referred to as the transport-triggered architecture (TTA).
    The work from Lama \textit{et al.} \cite{ttagpu} also demonstrated a framework that takes the advantage of TTA in graphic processing unit (GPU), and this idea is a potential alternative for the DSP design.
    \\\indent
    On the other hand, one may notices that none of the above implementations adopts HSA \cite{systemspec}, which is a promising standard for embedded DSP platforms.
    So far, most of studies on HSA focus on the integration of CPU and GPU. 
    \cite{hsaemu} presented a full system emulator for HSA platforms that include CPUs and GPUs.
    Beyond emulation, the study on a cycle-accurate HSA simulator that integrates QEMU \cite{qemu} and GPGPU-sim \cite{gpgpusim} is in full swing.
    To the best of our knowledge, little or no research has been conducted on applying the standard to DSP platforms.
    \\\indent
    Nevertheless, we also noticed that some of studies aiming at accelerating GPU can be applicable to HSA with DSPs.
    Hsu \textit{et.al.} proposed a hardware-based memory layout controller for heterogeneous systems~\cite{sc}.
    The controller optimizes irregular data structures from CPU and fit them to the memory footprint of GPU on-the-flight of memory transfer.
    The data locality in GPU is improved significantly as well as the latency of data layout conversion is hidden.
    Such a data conversion mechanism is also crucial to the integration of DeAr DSP and HSA, 
    and thus it will be presented in Section~\ref{sec:integration} with reference to \cite{sc}.
